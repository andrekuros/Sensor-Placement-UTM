% Section 4: Methodology

\section{Enhanced MUSCAT Methodology}

Our methodology extends the four-phase MUSCAT approach with three key enhancements: GA-based optimization, OSM urban environment integration, and deterministic ray-tracing.

\subsection{Phase 1: Environment Setup}

\subsubsection{Urban Data Acquisition}

Unlike MUSCAT's terrain-focused approach using DTED maps, we employ OpenStreetMap (OSM) to obtain real urban building data:

\begin{algorithmic}
\STATE Input: Latitude, Longitude, Radius
\STATE Download buildings from OSM API
\STATE Extract geometries (Polygon/MultiPolygon)
\STATE Estimate heights (from metadata or heuristics)
\STATE Reproject to UTM coordinates (meters)
\STATE Translate to local origin
\STATE Output: Buildings GeoJSON
\end{algorithmic}

This approach enables analysis of any urban area worldwide with freely available, crowd-sourced data that includes building footprints and attributes.

\subsubsection{3D Voxelization}

We discretize the urban environment into a 3D occupancy grid:

\begin{equation}
\text{Grid}(i,j,k) = \begin{cases}
1 & \text{if voxel occupied by building} \\
0 & \text{if free space}
\end{cases}
\end{equation}

where $(i,j,k)$ are voxel indices and resolution $r$ (typically 10-20m) is user-configurable. This voxelization enables efficient ray-tracing and LoS calculations.

\subsection{Phase 2: Sensor Attributes and ROI Definition}

\subsubsection{Sensor Models}

We implement four sensor types with physical parameters:

\textbf{Radar}: Active sensor with parameters from Table \ref{tab:sensor_params}
\begin{equation}
P_D^{radar} = f(P_t, G, \sigma_{RCS}, R, \lambda)
\end{equation}

\textbf{RF Passive (RID equivalent)}: Receives UAV broadcasts
\begin{equation}
P_D^{RF} = f(ERP, Sensitivity, R)
\end{equation}

\textbf{Electro-Optical (EO)}: Visual/IR detection

\textbf{Acoustic}: Sound-based detection

\subsubsection{Sensor Location Generation}

For real scenarios, sensor locations are automatically generated on a grid pattern while avoiding building interiors:

\begin{algorithmic}
\STATE Generate grid with spacing $s$ meters
\FOR{each grid point $(x, y)$}
  \IF{point not inside any building}
    \STATE Add location with height $h \sim U(25, 35)$ m
  \ENDIF
\ENDFOR
\end{algorithmic}

\subsection{Phase 3: Coverage Computation}

\subsubsection{Deterministic Ray-Tracing}

We replace statistical LoS models with deterministic ray-tracing:

\begin{algorithmic}
\STATE Input: Sensor location $\mathbf{s}$, Target location $\mathbf{t}$
\STATE Discretize 3D line between $\mathbf{s}$ and $\mathbf{t}$
\STATE Sample voxels along the ray
\IF{any voxel is occupied}
  \STATE $P_{LoS} = 0.0$ (blocked)
\ELSE
  \STATE $P_{LoS} = 1.0$ (clear)
\ENDIF
\end{algorithmic}

This provides exact LoS determination rather than probabilistic estimates, improving accuracy for urban scenarios.

\subsubsection{Network Detection Probability}

For multiple sensors, we employ soft voting fusion:

\begin{equation}
\pnet = 1 - \prod_{i=1}^{N_s} (1 - P_{D,i})
\label{eq:pnet}
\end{equation}

where $P_{D,i}$ is the detection probability of the $i$-th sensor.

\subsubsection{Redundancy Metric}

System redundancy is calculated as:

\begin{equation}
\rnet = \frac{1}{N_{free}} \sum_{(i,j,k) \in \text{Free}} \frac{\sum_s P_{D,s}(i,j,k)}{N_s}
\label{eq:redundancy}
\end{equation}

where $N_{free}$ is the number of unoccupied voxels and $N_s$ is the number of active sensors.

\subsection{Phase 4: Genetic Algorithm Optimization}

\subsubsection{GA Implementation}

We employ DEAP (Distributed Evolutionary Algorithms in Python) with the following operators:

\textbf{Selection}: Tournament selection with tournament size 3

\textbf{Crossover}: Two-point crossover preserving gene structure

\textbf{Mutation}: 
\begin{itemize}
\item Location mutation: Change $l_i$ to random valid location
\item Type mutation: Change $t_i$ to different sensor type
\item Activation mutation: Toggle $a_i$ between 0 and 1
\end{itemize}

\textbf{Fitness Evaluation}: Equation \ref{eq:fitness} with parallelized computation across available CPU cores.

\subsubsection{Convergence and Elitism}

We maintain the best individual across generations (elitism) and track fitness statistics to monitor convergence. Typical runs achieve stable solutions within 50-100 generations for population sizes of 30-50.

\subsection{Multi-Objective Optimization with NSGA-II}

The sensor placement problem inherently involves conflicting objectives that cannot be simultaneously optimized. To address this, we employ the Non-dominated Sorting Genetic Algorithm II (NSGA-II) \cite{deb2002fast} for multi-objective optimization.

\subsubsection{Problem Formulation}

Given a sensor configuration $X$, we define three objectives:

\begin{align}
\text{maximize} \quad & f_1(X) = M_c(X) \quad \text{(Coverage Index)} \label{eq:obj1}\\
\text{maximize} \quad & f_2(X) = \bar{R}(X) \quad \text{(Mean Redundancy)} \label{eq:obj2}\\
\text{minimize} \quad & f_3(X) = C(X) \quad \text{(Total Cost)} \label{eq:obj3}
\end{align}

A solution $X_1$ \textit{dominates} $X_2$ if $X_1$ is no worse in all objectives and strictly better in at least one. The \textit{Pareto-optimal front} is the set of all non-dominated solutions.

\subsubsection{NSGA-II Algorithm}

NSGA-II discovers the Pareto front through:

\begin{enumerate}
\item \textbf{Non-dominated Sorting}: Population is partitioned into fronts $F_1, F_2, \ldots$ where $F_1$ contains all non-dominated individuals, $F_2$ contains individuals dominated only by $F_1$, etc.

\item \textbf{Crowding Distance}: Within each front, individuals are assigned a crowding distance measuring solution density. Larger distances indicate less crowded regions, promoting solution diversity along the Pareto front.

\item \textbf{Selection}: Tournament selection favors individuals with better (lower) front rank and, for equal rank, larger crowding distance—balancing convergence toward Pareto front with diversity maintenance.

\item \textbf{Elitism}: The best non-dominated solutions are preserved across generations, ensuring Pareto front quality monotonically improves.
\end{enumerate}

This approach provides decision-makers with a continuum of optimal trade-offs rather than a single prescribed configuration, enabling context-specific infrastructure selection based on operational requirements, budget constraints, and risk tolerance.

\subsection{Computational Complexity Analysis}

\textbf{MUSCAT Exhaustive Search}:
\begin{equation}
O(\text{MUSCAT}) = O\left(\sum_{k=1}^{n} \binom{n}{k}\right) = O(2^n)
\end{equation}

\textbf{Our GA Approach}:
\begin{equation}
O(\text{GA}) = O(g \cdot p \cdot e)
\end{equation}
where $g$ is generations, $p$ is population size, and $e$ is evaluation complexity.

For $n=30$ locations:
\begin{itemize}
\item MUSCAT: $2^{30} \approx 10^9$ evaluations (intractable)
\item GA: $50 \times 30 \times e = 1,500 \cdot e$ evaluations (feasible)
\end{itemize}

This represents approximately 6 orders of magnitude reduction in computational requirements, enabling optimization of much larger networks.


