% Section 3: Problem Formulation

\section{Problem Formulation}

\subsection{Multi-Sensor Optimization Problem}

Following the MUSCAT formulation \cite{muscat2023}, we define the sensor placement problem as a resource allocation optimization. The objective is to determine the optimal number, combination, and placement of sensors in a given Region of Interest (ROI) to maximize system performance under resource constraints.

\subsubsection{Sets and Indices}

\begin{itemize}
\item $T$: Set of sensor types, indexed by $t \in T$
\item $L$: Set of potential sensor locations, indexed by $l \in L$
\item $U$: Set of UAV types (characterized by RCS), indexed by $u \in U$
\item $H$: Set of flight altitudes, indexed by $h \in H$
\end{itemize}

\subsubsection{Parameters}

\begin{itemize}
\item $AB$: Available budget (cost limit)
\item $C_t$: Unit cost of sensor type $t$
\item $SC_{tluh}$: Coverage provided by sensor type $t$ at location $l$ detecting UAV $u$ at altitude $h$
\item $\sigma$: Conversion parameter for objective function
\end{itemize}

\subsubsection{Decision Variables}

\begin{equation}
x_{tluh} = \begin{cases}
1 & \text{if sensor } t \text{ is placed at location } l \\
0 & \text{otherwise}
\end{cases}
\end{equation}

\subsubsection{Objective Function}

Maximize total coverage minus cost penalty:

\begin{equation}
\max \sum_{t \in T} \sum_{l \in L} \sum_{u \in U} \sum_{h \in H} SC_{tluh} \cdot x_{tluh} - \sigma \sum_{t \in T} \sum_{l \in L} C_t \cdot x_{tluh}
\label{eq:objective}
\end{equation}

\subsubsection{Constraints}

\begin{equation}
\sum_{t \in T} \sum_{l \in L} \sum_{u \in U} \sum_{h \in H} C_t \cdot x_{tluh} \leq AB
\label{eq:budget}
\end{equation}

\begin{equation}
x_{tluh} \in \{0, 1\} \quad \forall t \in T, l \in L, u \in U, h \in H
\label{eq:binary}
\end{equation}

\subsection{Genetic Algorithm Reformulation}

For GA implementation, we reformulate the decision variables as a chromosome representation:

\begin{equation}
\textbf{X} = [(l_1, t_1, a_1), (l_2, t_2, a_2), ..., (l_n, t_n, a_n)]
\label{eq:chromosome}
\end{equation}

where each gene $(l_i, t_i, a_i)$ represents:
\begin{itemize}
\item $l_i \in L$: sensor location index
\item $t_i \in T$: sensor type
\item $a_i \in \{0, 1\}$: activation status
\end{itemize}

The fitness function becomes:

\begin{equation}
F(\textbf{X}) = w_1 \cdot \pnet(\textbf{X}) + w_2 \cdot \rnet(\textbf{X}) - w_3 \cdot C(\textbf{X})
\label{eq:fitness}
\end{equation}

where:
\begin{itemize}
\item $\pnet(\textbf{X})$: Average network detection probability
\item $\rnet(\textbf{X})$: Average redundancy
\item $C(\textbf{X})$: Total cost
\item $w_1, w_2, w_3$: Weight parameters
\end{itemize}

\subsection{MUSCAT Metrics}

To maintain compatibility with the original framework, we implement the following metrics:

\textbf{Coverage Index} (Equation 6 from \cite{muscat2023}):
\begin{equation}
\mc = \frac{n}{N}
\label{eq:coverage_index}
\end{equation}
where $n$ is the number of grid cells with detection probability $P_D \geq$ threshold and $N$ is the total number of cells.

\textbf{Gap Index} (Equation 7 from \cite{muscat2023}):
\begin{equation}
\mg = 1 - \mc
\label{eq:gap_index}
\end{equation}

\textbf{Cost-Effectiveness} (Equation 8 from \cite{muscat2023}):
\begin{equation}
CA = \frac{\sum_{t \in T} C_t \cdot x_t}{\mc}
\label{eq:cost_effectiveness}
\end{equation}

\subsection{Requirements}

Following the MUSCAT baseline study \cite{muscat2023}, we adopt the following requirements for corridor coverage:

\begin{itemize}
\item $\mc > 0.95$ (coverage exceeds 95\%)
\item Overlap $> 0.55$ (multi-sensor redundancy exceeds 55\%)
\end{itemize}

These requirements ensure comprehensive monitoring and system resilience against sensor failures or cyber-attacks.


